%------------------------------  ------------------------------
%package list
\documentclass{article}
\usepackage[top=3cm, bottom=3cm, outer=3cm, inner=3cm]{geometry}
\usepackage{multicol}
\usepackage{graphicx}
\usepackage{url}
%\usepackage{cite}
\usepackage{hyperref}
\usepackage{array}
%\usepackage{multicol}
\newcolumntype{x}[1]{>{\centering\arraybackslash\hspace{0pt}}p{#1}}
\usepackage{natbib}
\usepackage{pdfpages}
\usepackage{multirow}
\usepackage[normalem]{ulem}
\useunder{\uline}{\ul}{}
\usepackage{svg}
\usepackage{xcolor}
\usepackage{listings}
\lstdefinestyle{ascii-tree}{
    literate={├}{|}1 {─}{--}1 {└}{+}1
  }
\lstset{basicstyle=\ttfamily,
  showstringspaces=false,
  commentstyle=\color{red},
  keywordstyle=\color{blue}
}
%\usepackage{booktabs}
\usepackage{caption}
\usepackage{subcaption}
\usepackage{float}
\usepackage{array}

\newcolumntype{M}[1]{>{\centering\arraybackslash}m{#1}}
\newcolumntype{N}{@{}m{0pt}@{}}

%------------------------------ ÍTEMS ------------------------------

\newcommand{\itemEmail}{hchoquehuancaz@unsa.edu.pe}
\newcommand{\itemStudent}{Hernan Andy Choquehuanca Zapana}
\newcommand{\itemCourse}{Fundamentos de la Programación II}
\newcommand{\itemCourseCode}{20232191}
\newcommand{\itemSemester}{II}
\newcommand{\itemUniversity}{Universidad Nacional de San Agustín de Arequipa}
\newcommand{\itemFaculty}{Facultad de Ingeniería de Producción y Servicios}
\newcommand{\itemDepartment}{Departamento Académico de Ingeniería de Sistemas e Informática}
\newcommand{\itemSchool}{Escuela Profesional de Ingeniería de Sistemas}
\newcommand{\itemAcademic}{2023 - B}
\newcommand{\itemInput}{Del 20 Septiembre 2023}
\newcommand{\itemOutput}{Al 25 Septiembre 2023}
\newcommand{\itemPracticeNumber}{03}
\newcommand{\itemTheme}{Arreglos de Objetos}

%------------------------------  ------------------------------

\usepackage[english,spanish]{babel}
\usepackage[utf8]{inputenc}
\AtBeginDocument{\selectlanguage{Spanish}}
\renewcommand{\figurename}{Figura}
\renewcommand{\refname}{Referencias}
\renewcommand{\tablename}{Tabla} %esto no funciona cuando se usa babel
\AtBeginDocument{%
	\renewcommand\tablename{Tabla}
}

\usepackage{fancyhdr}
\pagestyle{fancy}
\fancyhf{}
\setlength{\headheight}{30pt}
\renewcommand{\headrulewidth}{1pt}
\renewcommand{\footrulewidth}{1pt}
\fancyhead[L]{\raisebox{-0.2\height}{\includegraphics[width=3cm]{img/logo_episunsa.png}}}
\fancyhead[C]{\fontsize{7}{7}\selectfont	\itemUniversity \\ \itemFaculty \\ \itemDepartment \\ \itemSchool \\ \textbf{\itemCourse}}
\fancyhead[R]{\raisebox{-0.2\height}{\includegraphics[width=1.2cm]{img/logo_abet}}}
\fancyfoot[L]{Estudiante Hernan Choquehuanca Zapana}
\fancyfoot[R]{\itemCourse}
\fancyfoot[C]{Página \thepage}

% para el codigo fuente
\usepackage{listings}
\usepackage{color, colortbl}
\definecolor{dkgreen}{rgb}{0,0.6,0}
\definecolor{gray}{rgb}{0.5,0.5,0.5}
\definecolor{mauve}{rgb}{0.58,0,0.82}
\definecolor{codebackground}{rgb}{0.95, 0.95, 0.92}
\definecolor{tablebackground}{rgb}{0.8, 0, 0}

\lstset{frame=tb,
	language=bash,
	aboveskip=3mm,
	belowskip=3mm,
	showstringspaces=false,
	columns=flexible,
	basicstyle={\small\ttfamily},
	numbers=none,
	numberstyle=\tiny\color{gray},
	keywordstyle=\color{blue},
	commentstyle=\color{dkgreen},
	stringstyle=\color{mauve},
	breaklines=true,
	breakatwhitespace=true,
	tabsize=3,
	backgroundcolor= \color{codebackground},
}

%------------------------------ INICIO DEL DOCUMENTO------------------------------

\begin{document}
	
	\vspace*{10px}
	
	\begin{center}	
		\fontsize{17}{17} \textbf{ Informe de Laboratorio \itemPracticeNumber}
	\end{center}
	\centerline{\textbf{\Large Tema: \itemTheme}}
	%\vspace*{0.5cm}	

	\begin{flushright}
		\begin{tabular}{|M{2.5cm}|N|}
			\hline 
			\rowcolor{tablebackground}
			\color{white} \textbf{Nota}  \\
			\hline 
			     \\[30pt]
			\hline 			
		\end{tabular}
	\end{flushright}	

	\begin{table}[H]
		\begin{tabular}{|x{4.7cm}|x{4.8cm}|x{4.8cm}|}
			\hline 
			\rowcolor{tablebackground}
			\color{white} \textbf{Estudiante} & \color{white}\textbf{Escuela}  & \color{white}\textbf{Asignatura}   \\
			\hline 
			{\itemStudent \par \itemEmail} & \itemSchool & {\itemCourse \par Semestre: \itemSemester \par Código: \itemCourseCode}     \\
			\hline 			
		\end{tabular}
	\end{table}		
	
	\begin{table}[H]
		\begin{tabular}{|x{4.7cm}|x{4.8cm}|x{4.8cm}|}
			\hline 
			\rowcolor{tablebackground}
			\color{white}\textbf{Laboratorio} & \color{white}\textbf{Tema}  & \color{white}\textbf{Duración}   \\
			\hline 
			\itemPracticeNumber & \itemTheme & 02 horas   \\
			\hline 
		\end{tabular}
	\end{table}
	
	\begin{table}[H]
		\begin{tabular}{|x{4.7cm}|x{4.8cm}|x{4.8cm}|}
			\hline 
			\rowcolor{tablebackground}
			\color{white}\textbf{Semestre académico} & \color{white}\textbf{Fecha de inicio}  & \color{white}\textbf{Fecha de entrega}   \\
			\hline 
			\itemAcademic & \itemInput &  \itemOutput  \\
			\hline 
		\end{tabular}
	\end{table}

%------------------------------ ACTIVIDADES (TAREA) ------------------------------

	\section{Tarea}
	\begin{itemize}		
		\item Analice, complete y pruebe el Código de la clase DemoBatalla 
        \item Solucionar la Actividad 4 de la Práctica 1 pero usando arreglo de objetos
        \item Solucionar la Actividad 5 de la Práctica 1 pero usando arreglos de objetos
        
		\item Utilizar Git para evidenciar su trabajo.

	\end{itemize}
		
	\section{Equipos, materiales y temas utilizados}
	\begin{itemize}
		\item Sistema Operativo Windows 11 Pro 22H2 64 bits.
		\item VIM 9.0.
		\item Visual Studio Code
		\item Git 2.41.1.
		\item Cuenta en GitHub con el correo institucional.
            \item Variables Simples
		\item Arreglos de Objetos
        \item Métodos
	\end{itemize}
	
	\section{URL de Repositorio Github}
	\begin{itemize}
		\item URL del Repositorio GitHub para clonar o recuperar.
        \item \url{https://github.com/hernanchoquehuanca/fp2-23b.git}
		\item URL para el laboratorio 03 en el Repositorio GitHub.
		\item \url{https://github.com/hernanchoquehuanca/fp2-23b/tree/main/fase01/lab03}
	\end{itemize}
	
	\section{Actividades con el repositorio GitHub}
        
        
%-----------------------------------------------------------------------------------
%------------------------------------- ACTIVIDADES  --------------------------------
%-----------------------------------------------------------------------------------

% ACTIVIDAD 1
    \subsection{Commits}
    \subsubsection{Actividad 1 : Analice, complete y pruebe el Código de la clase DemoBatalla :}
    \begin{itemize}	
        \item Primero acomodamos y copiamos el código para luego empezar a completarlo
	\item El código fue el siguiente:
    \end{itemize}
    \lstinputlisting[language=Java, caption={DemoBatalla.java},numbers=left,]{src/DemoBatallaInicial.java}
    \lstinputlisting[language=Java, caption={Nave.java},numbers=left,]{src/NaveInicial.java}
    
    \begin{lstlisting}[language=bash,caption={Commit: Agregando las dos clases (DemoBatalla y Nave) para completar los métodos}][H]
		$ git add .
		$ git commit -m "Agregando las dos clases (DemoBatalla y Nave) para completar los métodos"			
		$ git push -u origin main
	\end{lstlisting}

    \begin{itemize}	
        \item Realizamos el método mostrarPorPuntos usando que mostrará aquellas naves menores o iguales a los puntos ingresados.
        \item Además creamos un método llamado mostrarNave, la cual nos servirá para los futuros métodos y este.
	\item El método fue el siguiente:
    \end{itemize}
    \lstinputlisting[language=Java, caption={DemoBatalla.java},numbers=left,]{src/m1.java}

    \begin{lstlisting}[language=bash,caption={Commit: Implementando el método mostrar por puntos, creando un nuevo método para mostrar naves (mostrarNave)}][H]
		$ git add .
		$ git commit -m "Implementando el método mostrar por puntos, creando un nuevo método para mostrar naves (mostrarNave)"			
		$ git push -u origin main
    \end{lstlisting}

    %%%
    
    \begin{itemize}	
        \item Implementando el método mostrar por nombre, haciendo uso de un bucle for para luego comparar todos los elementos de flota, mostrando con el método mostrarNaver a aquellos que coincidan con el nombre ingresado.
	\item El método fue el siguiente:
    \end{itemize}
    \lstinputlisting[language=Java, caption={DemoBatalla.java},numbers=left,]{src/m2.java}

    \begin{lstlisting}[language=bash,caption={Commit: Implementando el metodo mostrarPorNombre haciendo uso de un bucle for y el método mostrarNave}][H]
		$ git add .
		$ git commit -m "Implementando el método mostrarPorNombre haciendo uso de un bucle for y el metodo mostrarNave"
		$ git push -u origin main
    \end{lstlisting}
    
    %%%

    \begin{itemize}	
        \item Implementando el método mostrarMayorPuntos haciendo uso de variables auxiliares que permiten guardar datos enteros como el número máximo y su posición en el arreglo de naves.
	\item El método fue el siguiente:
    \end{itemize}
    \lstinputlisting[language=Java, caption={DemoBatalla.java},numbers=left,]{src/m3.java}

    \begin{lstlisting}[language=bash,caption={Commit: Implementando el método mostrarMayorPuntos haciendo uso de variables auxiliares para guardar valores de puntaje y posición}][H]
		$ git add .
		$ git commit -m "Implementando el método mostrarMayorPuntos haciendo uso de variables auxiliares para guardar valores de puntaje y posición"
		$ git push -u origin main
    \end{lstlisting}

    %%%

    \begin{itemize}	
        \item Implementando el método mostrarNaves simplemente usando un bucle for para recorrer por el arreglo y dentro utilizando el futuro método mostrarNave.
	\item El método fue el siguiente:
    \end{itemize}
    \lstinputlisting[language=Java, caption={DemoBatalla.java},numbers=left,]{src/m4.java}

    \begin{lstlisting}[language=bash,caption={Commit: Implementando el método mostrarNaves, utilizando el futuro método mostrarNave}][H]
		$ git add .
		$ git commit -m "Implementando el método mostrarNaves, utilizando el futuro metodo mostrarNave"
		$ git push -u origin main
    \end{lstlisting}

    %%%

    \begin{itemize}	
        \item Agregando comentarios a los métodos, para de alguna manera separarlos al imprimirlos dentro del main.
        \item Implementando el método desordenarFlota, utilizando Math.random para generar posiciones aleatorias.
        \item Para evitar las posiciones repetidas se agrego un bucle for para generar posiciones random hasta que se encuentre una vacía.
	\item El método fue el siguiente:
    \end{itemize}
    \lstinputlisting[language=Java, caption={DemoBatalla.java},numbers=left,]{src/m5.java}

    \begin{lstlisting}[language=bash,caption={Commit: Implementando el método para retornar un arreglo aleatoriamente desordenado}][H]
		$ git add .
		$ git commit -m "Implementando el método para retornar un arreglo aleatoriamente desordenado"
		$ git push -u origin main
    \end{lstlisting}

    %%%

    \begin{itemize}	
        \item Finalmente implementamos el método mostrarNave, el cual está presente en los demás métodos.
        \item El método imprimirá el nombre, estado y puntos.
	\item El método fue el siguiente:
    \end{itemize}
    \lstinputlisting[language=Java, caption={DemoBatalla.java},numbers=left,]{src/m6.java}

    \begin{lstlisting}[language=bash,caption={Commit: Implementando el método mostrarNave, el cual está presente en otros métodos}][H]
		$ git add .
		$ git commit -m "Implementando el método mostrarNave, el cual está presente en otros métodos"
		$ git push -u origin main
    \end{lstlisting}


    
%%% 
    \subsubsection{Actividad 2 : Solucionar la Actividad 4 de la Práctica 1 pero usando arreglo de objetos}
    \begin{itemize}	
        \item Primero acomodamos y copiamos el código para luego empezar a editarlo.
        \item Reajustando variables y ahora haciendo uso de la nueva clase Soldier.
	\item El código fue el siguiente:
    \end{itemize}
    \lstinputlisting[language=Java, caption={Ejercicio01.java},numbers=left,]{src/Ejercicio01.java}
    \lstinputlisting[language=Java, caption={Soldier01.java},numbers=left,]{src/Soldier01.java}
    
    \begin{lstlisting}[language=bash,caption={Commit: Ejercicio01 terminado, creando otra clase llamada Soldier para los objetos de soldados}][H]
		$ git add .
		$ git commit -m "Ejercicio01 terminado, creando otra clase llamada Soldier para los objetos de soldados"			
		$ git push -u origin main
	\end{lstlisting}

    %%%


    \subsubsection{Actividad 3 : Solucionar la Actividad 5 de la Práctica 1 pero usando arreglos de objetos}
    \begin{itemize}	
        \item Primero acomodamos y copiamos el código para luego empezar a completarlo
        \item Reajustando variables y ahora haciendo uso de la nueva clase Soldier.
	\item El código fue el siguiente:
    \end{itemize}
    \lstinputlisting[language=Java, caption={Ejercicio02.java},numbers=left,]{src/Ejercicio02.java}
    \lstinputlisting[language=Java, caption={Soldier02.java},numbers=left,]{src/Soldier02.java}
    
    \begin{lstlisting}[language=bash,caption={Commit: Reemplazando instanciaciones y demás se acomodó el código para que utilice arrays de Soldier en lugar de String}][H]
		$ git add .
		$ git commit -m "Reemplazando instanciaciones y demás se acomodó el código para que utilice arrays de Soldier en lugar de String"			
		$ git push -u origin main
	\end{lstlisting}

    
%-----------------------------------------------------------------------------------
%------------------------------ ESTRUCTURA DE LABORATORIO --------------------------
%-----------------------------------------------------------------------------------
    \newpage
	\subsection{Estructura de laboratorio 03}
    \begin{itemize}	
		\item El contenido que se entrega en este laboratorio es el siguiente:
	\end{itemize}
	
\begin{lstlisting}[style=ascii-tree]

lab03
│   DemoBatalla.java
│   Ejercicio01.java
│   Ejercicio02.java
│   Nave.java
│   Soldier.java
│
└───latex
    │   Informe_Lab03.pdf
    │   Informe_Lab03.tex
    │
    ├───img
    │       logo_abet.png
    │       logo_episunsa.png
    │       logo_unsa.jpg
    │       prueba01.png
    │
    └───src
            DemoBatallaInicial.java
            Ejercicio01.java
            Ejercicio02.java
            m1.java
            m2.java
            m3.java
            m4.java
            m5.java
            m6.java
            NaveInicial.java
            Soldier01.java
            Soldier02.java

\end{lstlisting}    

	\section{\textcolor{red}{Rúbricas}}
	
	\subsection{\textcolor{red}{Entregable Informe}}
	\begin{table}[H]
		\caption{Tipo de Informe}
		\setlength{\tabcolsep}{0.5em} % for the horizontal padding
		{\renewcommand{\arraystretch}{1.5}% for the vertical padding
		\begin{tabular}{|p{3cm}|p{12cm}|}
			\hline
			\multicolumn{2}{|c|}{\textbf{\textcolor{red}{Informe}}}  \\
			\hline 
			\textbf{\textcolor{red}{Latex}} & \textcolor{blue}{El informe está en formato PDF desde Latex,  con un formato limpio (buena presentación) y facil de leer.}   \\ 
			\hline 
			
			
		\end{tabular}
	}
	\end{table}
	
	\clearpage
%-----------------------------------------------------------------------------------
%------------------------------ RÚBRICA DE EVALUACIÓN ------------------------------
%-----------------------------------------------------------------------------------
 
	\subsection{\textcolor{red}{Rúbrica para el contenido del Informe y demostración}}
	\begin{itemize}			
		\item El alumno debe marcar o dejar en blanco en celdas de la columna \textbf{Checklist} si cumplio con el ítem correspondiente.
		\item Si un alumno supera la fecha de entrega,  su calificación será sobre la nota mínima aprobada, siempre y cuando cumpla con todos lo items.
		\item El alumno debe autocalificarse en la columna \textbf{Estudiante} de acuerdo a la siguiente tabla:
	
		\begin{table}[ht]
			\caption{Niveles de desempeño}
			\begin{center}
			\begin{tabular}{ccccc}
    			\hline
    			 & \multicolumn{4}{c}{Nivel}\\
    			\cline{1-5}
    			\textbf{Puntos} & Insatisfactorio 25\%& En Proceso 50\% & Satisfactorio 75\% & Sobresaliente 100\%\\
    			\textbf{2.0}&0.5&1.0&1.5&2.0\\
    			\textbf{4.0}&1.0&2.0&3.0&4.0\\
    		\hline
			\end{tabular}
		\end{center}
	\end{table}	
	
	\end{itemize}
	
	\begin{table}[H]
		\caption{Rúbrica para contenido del Informe y demostración}
		\setlength{\tabcolsep}{0.5em} % for the horizontal padding
		{\renewcommand{\arraystretch}{1.5}% for the vertical padding
		%\begin{center}
		\begin{tabular}{|p{2.7cm}|p{7cm}|x{1.3cm}|p{1.2cm}|p{1.5cm}|p{1.1cm}|}
			\hline
    		\multicolumn{2}{|c|}{Contenido y demostración} & Puntos & Checklist & Estudiante & Profesor\\
			\hline
			\textbf{1. GitHub} & Hay enlace URL activo del directorio para el  laboratorio hacia su repositorio GitHub con código fuente terminado y fácil de revisar. &2 &X &2 & \\ 
			\hline
			\textbf{2. Commits} &  Hay capturas de pantalla de los commits más importantes con sus explicaciones detalladas. (El profesor puede preguntar para refrendar calificación). &4 &X &3 & \\ 
			\hline 
			\textbf{3. Código fuente} &  Hay porciones de código fuente importantes con numeración y explicaciones detalladas de sus funciones. &2 &X &2 & \\ 
			\hline 
			\textbf{4. Ejecución} & Se incluyen ejecuciones/pruebas del código fuente  explicadas gradualmente. &2 & & & \\ 
			\hline			
			\textbf{5. Pregunta} & Se responde con completitud a la pregunta formulada en la tarea.  (El profesor puede preguntar para refrendar calificación).  &2 &X &2 & \\ 
			\hline	
			\textbf{6. Fechas} & Las fechas de modificación del código fuente estan dentro de los plazos de fecha de entrega establecidos. &2 &X &2 & \\ 
			\hline 
			\textbf{7. Ortografía} & El documento no muestra errores ortográficos. &2 &X &2 & \\ 
			\hline 
			\textbf{8. Madurez} & El Informe muestra de manera general una evolución de la madurez del código fuente,  explicaciones puntuales pero precisas y un acabado impecable.   (El profesor puede preguntar para refrendar calificación).  &4 &X &2 & \\ 
			\hline
			\multicolumn{2}{|c|}{\textbf{Total}} &20 & &15 & \\ 
			\hline
		\end{tabular}
		%\end{center}
		%\label{tab:multicol}
		}
	\end{table}
	
\clearpage

%------------------------------ REFERENCIAS ------------------------------

\section{Referencias}
\begin{itemize}			
    \item \url{https://docs.oracle.com/javase/tutorial/java/nutsandbolts/variables.html}
    \item \url{https://docs.oracle.com/javase/8/docs/api/java/util/Arrays.html}
    \item \url{https://docs.oracle.com/javase/tutorial/java/javaOO/methods.html}
\end{itemize}	
	
%\clearpage
%\bibliographystyle{apalike}
%\bibliographystyle{IEEEtranN}
%\bibliography{bibliography}
			
\end{document}