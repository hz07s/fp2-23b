%------------------------------  ------------------------------
%package list
\documentclass{article}
\usepackage[top=3cm, bottom=3cm, outer=3cm, inner=3cm]{geometry}
\usepackage{multicol}
\usepackage{graphicx}
\usepackage{url}
%\usepackage{cite}
\usepackage{hyperref}
\usepackage{array}
%\usepackage{multicol}
\newcolumntype{x}[1]{>{\centering\arraybackslash\hspace{0pt}}p{#1}}
\usepackage{natbib}
\usepackage{pdfpages}
\usepackage{multirow}
\usepackage[normalem]{ulem}
\useunder{\uline}{\ul}{}
\usepackage{svg}
\usepackage{xcolor}
\usepackage{listings}
\lstdefinestyle{ascii-tree}{
    literate={├}{|}1 {─}{--}1 {└}{+}1
  }
\lstset{basicstyle=\ttfamily,
  showstringspaces=false,
  commentstyle=\color{red},
  keywordstyle=\color{blue}
}
%\usepackage{booktabs}
\usepackage{caption}
\usepackage{subcaption}
\usepackage{float}
\usepackage{array}

\newcolumntype{M}[1]{>{\centering\arraybackslash}m{#1}}
\newcolumntype{N}{@{}m{0pt}@{}}

%------------------------------ ÍTEMS ------------------------------

\newcommand{\itemEmail}{hchoquehuancaz@unsa.edu.pe}
\newcommand{\itemStudent}{Hernan Andy Choquehuanca Zapana}
\newcommand{\itemCourse}{Fundamentos de la Programación II}
\newcommand{\itemCourseCode}{20232191}
\newcommand{\itemSemester}{II}
\newcommand{\itemUniversity}{Universidad Nacional de San Agustín de Arequipa}
\newcommand{\itemFaculty}{Facultad de Ingeniería de Producción y Servicios}
\newcommand{\itemDepartment}{Departamento Académico de Ingeniería de Sistemas e Informática}
\newcommand{\itemSchool}{Escuela Profesional de Ingeniería de Sistemas}
\newcommand{\itemAcademic}{2023 - B}
\newcommand{\itemInput}{Del 18 Septiembre 2023}
\newcommand{\itemOutput}{Al 20 Septiembre 2023}
\newcommand{\itemPracticeNumber}{01}
\newcommand{\itemTheme}{Arreglos Estándar}

%------------------------------  ------------------------------

\usepackage[english,spanish]{babel}
\usepackage[utf8]{inputenc}
\AtBeginDocument{\selectlanguage{Spanish}}
\renewcommand{\figurename}{Figura}
\renewcommand{\refname}{Referencias}
\renewcommand{\tablename}{Tabla} %esto no funciona cuando se usa babel
\AtBeginDocument{%
	\renewcommand\tablename{Tabla}
}

\usepackage{fancyhdr}
\pagestyle{fancy}
\fancyhf{}
\setlength{\headheight}{30pt}
\renewcommand{\headrulewidth}{1pt}
\renewcommand{\footrulewidth}{1pt}
\fancyhead[L]{\raisebox{-0.2\height}{\includegraphics[width=3cm]{img/logo_episunsa.png}}}
\fancyhead[C]{\fontsize{7}{7}\selectfont	\itemUniversity \\ \itemFaculty \\ \itemDepartment \\ \itemSchool \\ \textbf{\itemCourse}}
\fancyhead[R]{\raisebox{-0.2\height}{\includegraphics[width=1.2cm]{img/logo_abet}}}
\fancyfoot[L]{Estudiante Hernan Choquehuanca Zapana}
\fancyfoot[R]{\itemCourse}
\fancyfoot[C]{Página \thepage}

% para el codigo fuente
\usepackage{listings}
\usepackage{color, colortbl}
\definecolor{dkgreen}{rgb}{0,0.6,0}
\definecolor{gray}{rgb}{0.5,0.5,0.5}
\definecolor{mauve}{rgb}{0.58,0,0.82}
\definecolor{codebackground}{rgb}{0.95, 0.95, 0.92}
\definecolor{tablebackground}{rgb}{0.8, 0, 0}

\lstset{frame=tb,
	language=bash,
	aboveskip=3mm,
	belowskip=3mm,
	showstringspaces=false,
	columns=flexible,
	basicstyle={\small\ttfamily},
	numbers=none,
	numberstyle=\tiny\color{gray},
	keywordstyle=\color{blue},
	commentstyle=\color{dkgreen},
	stringstyle=\color{mauve},
	breaklines=true,
	breakatwhitespace=true,
	tabsize=3,
	backgroundcolor= \color{codebackground},
}

%------------------------------ INICIO DEL DOCUMENTO------------------------------

\begin{document}
	
	\vspace*{10px}
	
	\begin{center}	
		\fontsize{17}{17} \textbf{ Informe de Laboratorio \itemPracticeNumber}
	\end{center}
	\centerline{\textbf{\Large Tema: \itemTheme}}
	%\vspace*{0.5cm}	

	\begin{flushright}
		\begin{tabular}{|M{2.5cm}|N|}
			\hline 
			\rowcolor{tablebackground}
			\color{white} \textbf{Nota}  \\
			\hline 
			     \\[30pt]
			\hline 			
		\end{tabular}
	\end{flushright}	

	\begin{table}[H]
		\begin{tabular}{|x{4.7cm}|x{4.8cm}|x{4.8cm}|}
			\hline 
			\rowcolor{tablebackground}
			\color{white} \textbf{Estudiante} & \color{white}\textbf{Escuela}  & \color{white}\textbf{Asignatura}   \\
			\hline 
			{\itemStudent \par \itemEmail} & \itemSchool & {\itemCourse \par Semestre: \itemSemester \par Código: \itemCourseCode}     \\
			\hline 			
		\end{tabular}
	\end{table}		
	
	\begin{table}[H]
		\begin{tabular}{|x{4.7cm}|x{4.8cm}|x{4.8cm}|}
			\hline 
			\rowcolor{tablebackground}
			\color{white}\textbf{Laboratorio} & \color{white}\textbf{Tema}  & \color{white}\textbf{Duración}   \\
			\hline 
			\itemPracticeNumber & \itemTheme & 02 horas   \\
			\hline 
		\end{tabular}
	\end{table}
	
	\begin{table}[H]
		\begin{tabular}{|x{4.7cm}|x{4.8cm}|x{4.8cm}|}
			\hline 
			\rowcolor{tablebackground}
			\color{white}\textbf{Semestre académico} & \color{white}\textbf{Fecha de inicio}  & \color{white}\textbf{Fecha de entrega}   \\
			\hline 
			\itemAcademic & \itemInput &  \itemOutput  \\
			\hline 
		\end{tabular}
	\end{table}

%------------------------------ ACTIVIDADES (TAREA) ------------------------------

	\section{Tarea}
	\begin{itemize}		
		\item Escribir un programa donde se creen 5 soldados considerando sólo su nombre. Ingresar sus datos y después mostrarlos. 
        Restricción: se realizará considerando sólo los conocimientos que se tienen de FP1 y sin utilizar arreglos estándar, sólo usar variables simples.
        \item Escribir un programa donde se creen 5 soldados considerando su nombre y nivel de vida. Ingresar sus datos y después mostrarlos.
        Restricción: se realizará considerando sólo los conocimientos que se tienen de FP1 y sin utilizar arreglos estándar, sólo usar variables simples.
        \item Escribir un programa donde se creen 5 soldados considerando sólo su nombre. Ingresar sus datos y después mostrarlos.
        Restricción: aplicar arreglos estándar.
        \item Escribir un programa donde se creen 5 soldados considerando su nombre y nivel de vida Ingresar sus datos y después mostrarlos.
        Restricción: aplicar arreglos estándar. (Todavía no aplicar arreglo de objetos)
        \item Escribir un programa donde se creen 2 ejércitos, cada uno con un número aleatorio de soldados entre 1 y 5, considerando sólo su nombre. Sus datos se inicializan automáticamente con nombres tales como “Soldado0”, “Soldado1”, etc. Luego de crear los 2 ejércitos se deben mostrar los datos de todos los soldados de ambos ejércitos e indicar qué ejército fue el ganador. Restricción: aplicar arreglos estándar y métodos para inicializar los ejércitos, mostrar ejército y mostrar ejército ganador. La métrica a aplicar para indicar el ganador es el mayor número de soldados de cada ejército, puede haber empates. (Todavía no aplicar arreglo de objetos)
		\item Utilizar Git para evidenciar su trabajo.
		\item Enviar trabajo al profesor en un repositorio GitHub Privado, dándole permisos como colaborador.
	\end{itemize}
		
	\section{Equipos, materiales y temas utilizados}
	\begin{itemize}
		\item Sistema Operativo Windows 11 Pro 22H2 64 bits.
		\item VIM 9.0.
		\item OpenJDK 64-Bits 17.0.7.
		\item Git 2.41.1.
		\item Cuenta en GitHub con el correo institucional.
        \item Variables Simples
		\item Arreglos Estándar
	\end{itemize}
	
	\section{URL de Repositorio Github}
	\begin{itemize}
		\item URL del Repositorio GitHub para clonar o recuperar.
        \item \url{https://github.com/hernanchoquehuanca/fp2-23b.git}
		\item URL para el laboratorio 01 en el Repositorio GitHub.
		\item \url{https://github.com/hernanchoquehuanca/fp2-23b/tree/main/fase01/lab01}
	\end{itemize}
	
	\section{Actividades con el repositorio GitHub}
	
	\subsection{Creando e inicializando repositorio GitHub}
	\begin{itemize}	
		\item Como es el primer laboratorio se creo el repositorio GitHub.
		\item Se realizaron los siguientes comandos en la computadora:
	\end{itemize}	
		
	\begin{lstlisting}[language=bash,caption={Creando directorio de trabajo}][H]
		$ mkdir e:/fp2-23b/
	\end{lstlisting}
	\begin{lstlisting}[language=bash,caption={Dirijíéndonos al directorio de trabajo}][H]
		$ cd e:/fp2-23b/
	\end{lstlisting}	
	\begin{lstlisting}[language=bash,caption={Creando directorio para la primera fase}][H]
		$ mkdir e:/fp2-23b/fase01/
	\end{lstlisting}
	\begin{lstlisting}[language=bash,caption={Inicializando directorio para repositorio GitHub}][H]
		$ git init
		$ git config --global user.name "Hernan Andy Choquehuanca Zapana"
		$ git config --global user.email hchoquehuancaz@unsa.edu.pe
		$ git branch -M main
		$ git remote add origin https://github.com/hernanchoquehuanca/fp2-23b.git
   	$ git add Videojuego.java
		$ git commit -m "Agregando el saludo a VideoJuego.java con un mensaje de Bienvenida"
		$ git push -u origin main
	\end{lstlisting}
%-----------------------------------------------------------------------------------
%------------------------------------- ACTIVIDADES  --------------------------------
%-----------------------------------------------------------------------------------

% ACTIVIDAD 1

	\subsection{Commits}
    \subsubsection{Actividad 1 : Escribir un programa donde se creen 5 soldados considerando sólo su nombre. Ingresar sus datos y después mostrarlos. \\\\
    \color{red}Restricción: se realizará considerando sólo los conocimientos que se tienen de FP1 y sin utilizar arreglos estándar, sólo usar variables simples.}
	\begin{lstlisting}[language=bash,caption={Primer commit creando variables para los nombres de soldados}][H]
		$ mkdir lab01
		$ touch lab01/Ejercicio01.java
		$ git add .
		$ git commit -m "Ejercicio01 realizado con variables simples (Strings) para los datos de los 5 soldados"			
		$ git push -u origin main
	\end{lstlisting}
	
	\begin{itemize}	
        \item En el segundo commit se le agregó un mensaje al recibir e imprimir los nombres
		\item El código fue el siguiente:
	\end{itemize}

	\lstinputlisting[language=Java, caption={Ejercicio01.java},numbers=left,]{src/Ejercicio01v2.java}

% ACTIVIDAD 2

    \subsubsection{Actividad 2 : Escribir un programa donde se creen 5 soldados considerando su nombre y nivel de vida. Ingresar sus datos y después mostrarlos. \\\\ 
    \color{red}Restricción: se realizará considerando sólo los conocimientos que se tienen de FP1 y sin utilizar arreglos estándar, sólo usar variables simples.}

	\lstinputlisting[language=Java, caption={Ejercicio02.java},numbers=left,]{src/Ejercicio02.java}

    \begin{lstlisting}[language=bash,caption={Commit: Ejercicio02 realizado con variables simples (String e int) tanto para los nombres, como para la vida}][H]
		$ git add .
		$ git commit -m "Ejercicio02 realizado con variables simples (String e int) tanto para los nombres, como para la vida"			
		$ git push -u origin main
	\end{lstlisting}
    \begin{itemize}
        \item Utilizando el código Ejercicio01.java, se agregó variables enteras (int) para la vida de los soldados
    \end{itemize} 

% ACTIVIDAD 3

    \subsubsection{Actividad 3 : Escribir un programa donde se creen 5 soldados considerando sólo su nombre. Ingresar sus datos y después mostrarlos. \\\\
    \color{red}Restricción: Aplicar arreglos estándar.}
    
    \begin{itemize}
        \item  Utilizando arreglos estándar y un bucle for para recibir los datos y asignarles una posición en el arreglo "soldiers" según el índice "i"
    \end{itemize} 
    
    \lstinputlisting[language=Java, caption={Ejercicio03.java},numbers=left,]{src/Ejercicio03v1.java}

     \begin{lstlisting}[language=bash,caption={Commit: Ejercicio03 ahora usando arrays, utilizando un bucle for para recibir los nombres}][H]
		$ git add .
		$ git commit -m "Ejercicio03 ahora usando arrays, utilizando un bucle for para recibir los nombres"			
		$ git push -u origin main
	\end{lstlisting}
 
     \begin{itemize}
        \item  Para finalizar con ayuda de otro bucle for se imprimirán los nombres de los soldados ingresados
    \end{itemize} 
    
	\lstinputlisting[language=Java, caption={Ejercicio03.java},numbers=left,]{src/Ejercicio03v2.1.java}
    \begin{lstlisting}[language=bash,caption={Commit: Agregando un segundo bucle for para mostrar los datos (nombres)}][H]
		$ git add .
		$ git commit -m "Agregando un segundo bucle for para mostrar los datos (nombres)"			
		$ git push -u origin main
	\end{lstlisting}
 
    \begin{itemize}
        \item El código con los cambios ya mencionados es el siguiente: 
    \end{itemize} 
	\lstinputlisting[language=Java, caption={Ejercicio03.java},numbers=left,]{src/Ejercicio03v2.java}

% ACTIVIDAD 4

    \subsubsection{Actividad 4 : Escribir un programa donde se creen 5 soldados considerando su nombre y nivel de vida. Ingresar sus datos y después mostrarlos. \\\\
    \color{red}Restricción: Aplicar arreglos estándar. (Todavía no aplicar arreglo de objetos)}

    \begin{itemize}
        \item Utilizando el código de Ejercicio03.java 
        \item Se agregó un bucle for donde se recibiría la vida de los soldados en un arreglo llamado "healt"
    \end{itemize}
	\lstinputlisting[language=Java, caption={Ejercicio04.java},numbers=left,]{src/Ejercicio04v1.java}
    \begin{lstlisting}[language=bash,caption={Commit: Ejercicio04 usando los bucles del ejercicio anterior para los nombres y creando un bucle para recibir las vidas}][H]
		$ git add .
		$ git commit -m "Ejercicio04 usando los bucles del ejercicio anterior para los nombres y creando un bucle para recibir las vidas"	
		$ git push -u origin main
	\end{lstlisting}

    \begin{itemize}
        \item Finalmente se cambió la entrada de la vida de los soldados, colocándolos dentro del mismo bucle for que recibe los nombres
    \end{itemize}
    \lstinputlisting[language=Java, caption={Ejercicio04.java},numbers=left,]{src/Ejercicio04v2.java}
    \begin{lstlisting}[language=bash,caption={Commit: Recibiendo la vida de los soldados dentro del mismo bucle que los nombres para reducir código}][H]
		$ git add .
		$ git commit -m "Recibiendo la vida de los soldados dentro del mismo bucle que los nombres para reducir código"			
		$ git push -u origin main
	\end{lstlisting}

    \begin{itemize}
        \item El código con los cambios ya mencionados es el siguiente: 
    \end{itemize} 
	\lstinputlisting[language=Java, caption={Ejercicio04.java},numbers=left,]{src/Ejercicio04.java}

% ACTIVIDAD 5

    \subsubsection{Actividad 5 : Escribir un programa donde se creen 2 ejércitos, cada uno con un número aleatorio de soldados entre 1 y 5, considerando sólo su nombre. Sus datos se inicializan automáticamente con nombres tales como “Soldado0”, “Soldado1”, etc. Luego de crear los 2 ejércitos se deben mostrar los datos de todos los soldados de ambos ejércitos e indicar qué ejército fue el ganador. \\\\
    \color{red}Restricción: aplicar arreglos estándar y métodos para inicializar los ejércitos, mostrar ejército y mostrar ejército ganador. La métrica a aplicar para indicar el ganador es el mayor número de soldados de cada ejército, puede haber empates. (Todavía no aplicar arreglo de objetos)}

    \begin{itemize}
        \item Comenzando por el método que creará los ejércitos, este generará un entero n que indicará el número de soldados del ejército
    \end{itemize}    
    \lstinputlisting[language=Java, caption={Ejercicio05.java},numbers=left,]{src/Ejercicio05v1.java}

    \begin{itemize}
        \item Seguidamente implementamos el método para mostrar los ejércitos que recibe como parámetro un arreglo de Strings
    \end{itemize}    
    \lstinputlisting[language=Java, caption={Ejercicio05.java},numbers=left,]{src/Ejercicio05v2.java}

    \begin{itemize}
        \item Ahora añadimos un método que mostrará al ejército ganador, basándose en el tamaño del arreglo, ya que de esa manera se evalúa el ejército más grande
        \item En caso de que ambos ejércitos tengan el mismo tamaño se dará un empate, esto con la ayuda de condicionales (if - else if)
    \end{itemize}    
    \lstinputlisting[language=Java, caption={Ejercicio05.java},numbers=left,]{src/Ejercicio05v3.java}

    \begin{itemize}
        \item Finalmente ahora que ya tenemos los métodos, los usaremos en el main tal como se muestra a continuación :
    \end{itemize}    
    \lstinputlisting[language=Java, caption={Ejercicio05.java},]{src/Ejercicio05v4.java}

    \begin{itemize}
        \item En el último commit se subió todos los métodos unidos y siendo utilizandos en el main para darle funcionalidad al programa
        \item CADA UNO DE LOS MÉTODOS PRESENTADOS TIENEN SUS RESPECTIVOS COMMITS
    \end{itemize}

    \begin{lstlisting}[language=bash,caption={Commit: Agregando el main utilizando todos los metodos elaborados para terminar con el programa}][H]
		$ git add .
		$ git commit -m "Agregando el main utilizando todos los metodos elaborados para terminar con el programa"
		$ git push -u origin main
	\end{lstlisting}

    \newpage
    \begin{itemize}
        \item El código con todos los cambios ya mencionados es el siguiente: 
    \end{itemize}
	\lstinputlisting[language=Java, caption={Ejercicio05.java},numbers=left,]{src/Ejercicio05.java}
	
%-----------------------------------------------------------------------------------
%------------------------------ ESTRUCTURA DE LABORATORIO --------------------------
%-----------------------------------------------------------------------------------
    \newpage
	\subsection{Estructura de laboratorio 01}
	\begin{itemize}	
		\item El contenido que se entrega en este laboratorio es el siguiente:
	\end{itemize}
	
\begin{lstlisting}[style=ascii-tree]

lab01
|---|--- Ejercicio01.java
|---|--- Ejercicio02.java
|---|--- Ejercicio03.java
|---|--- Ejercicio04.java
|---|--- Ejercicio05.java
|---|--- VideoJuego.java
|--- latex
    |--- img
    |   |--- logo_abet.png
    |   |--- logo_episunsa.png
    |   |--- logo_unsa.jpg
    |--- Informe_lab01.pdf
    |--- Informe_lab01.tex
    |--- src
        |---|--- Ejercicio01v2.java
        |---|--- Ejercicio02.java
        |---|--- Ejercicio03v1.java
        |---|--- Ejercicio03v2.1.java
        |---|--- Ejercicio03v2.java
        |---|--- Ejercicio04.java
        |---|--- Ejercicio04v1.java
        |---|--- Ejercicio04v2.java
        |---|--- Ejercicio05.java
        |---|--- Ejercicio05v1.java
        |---|--- Ejercicio05v2.java
        |---|--- Ejercicio05v3.java
        |---|--- Ejercicio05v4.java
        |---|--- VideoJuego.java

\end{lstlisting}    

	\section{\textcolor{red}{Rúbricas}}
	
	\subsection{\textcolor{red}{Entregable Informe}}
	\begin{table}[H]
		\caption{Tipo de Informe}
		\setlength{\tabcolsep}{0.5em} % for the horizontal padding
		{\renewcommand{\arraystretch}{1.5}% for the vertical padding
		\begin{tabular}{|p{3cm}|p{12cm}|}
			\hline
			\multicolumn{2}{|c|}{\textbf{\textcolor{red}{Informe}}}  \\
			\hline 
			\textbf{\textcolor{red}{Latex}} & \textcolor{blue}{El informe está en formato PDF desde Latex,  con un formato limpio (buena presentación) y facil de leer.}   \\ 
			\hline 
			
			
		\end{tabular}
	}
	\end{table}
	
	\clearpage
%-----------------------------------------------------------------------------------
%------------------------------ RÚBRICA DE EVALUACIÓN ------------------------------
%-----------------------------------------------------------------------------------
 
	\subsection{\textcolor{red}{Rúbrica para el contenido del Informe y demostración}}
	\begin{itemize}			
		\item El alumno debe marcar o dejar en blanco en celdas de la columna \textbf{Checklist} si cumplio con el ítem correspondiente.
		\item Si un alumno supera la fecha de entrega,  su calificación será sobre la nota mínima aprobada, siempre y cuando cumpla con todos lo items.
		\item El alumno debe autocalificarse en la columna \textbf{Estudiante} de acuerdo a la siguiente tabla:
	
		\begin{table}[ht]
			\caption{Niveles de desempeño}
			\begin{center}
			\begin{tabular}{ccccc}
    			\hline
    			 & \multicolumn{4}{c}{Nivel}\\
    			\cline{1-5}
    			\textbf{Puntos} & Insatisfactorio 25\%& En Proceso 50\% & Satisfactorio 75\% & Sobresaliente 100\%\\
    			\textbf{2.0}&0.5&1.0&1.5&2.0\\
    			\textbf{4.0}&1.0&2.0&3.0&4.0\\
    		\hline
			\end{tabular}
		\end{center}
	\end{table}	
	
	\end{itemize}
	
	\begin{table}[H]
		\caption{Rúbrica para contenido del Informe y demostración}
		\setlength{\tabcolsep}{0.5em} % for the horizontal padding
		{\renewcommand{\arraystretch}{1.5}% for the vertical padding
		%\begin{center}
		\begin{tabular}{|p{2.7cm}|p{7cm}|x{1.3cm}|p{1.2cm}|p{1.5cm}|p{1.1cm}|}
			\hline
    		\multicolumn{2}{|c|}{Contenido y demostración} & Puntos & Checklist & Estudiante & Profesor\\
			\hline
			\textbf{1. GitHub} & Hay enlace URL activo del directorio para el  laboratorio hacia su repositorio GitHub con código fuente terminado y fácil de revisar. &2 &X &2 & \\ 
			\hline
			\textbf{2. Commits} &  Hay capturas de pantalla de los commits más importantes con sus explicaciones detalladas. (El profesor puede preguntar para refrendar calificación). &4 &X &3 & \\ 
			\hline 
			\textbf{3. Código fuente} &  Hay porciones de código fuente importantes con numeración y explicaciones detalladas de sus funciones. &2 &X &2 & \\ 
			\hline 
			\textbf{4. Ejecución} & Se incluyen ejecuciones/pruebas del código fuente  explicadas gradualmente. &2 & & & \\ 
			\hline			
			\textbf{5. Pregunta} & Se responde con completitud a la pregunta formulada en la tarea.  (El profesor puede preguntar para refrendar calificación).  &2 &X &2 & \\ 
			\hline	
			\textbf{6. Fechas} & Las fechas de modificación del código fuente estan dentro de los plazos de fecha de entrega establecidos. &2 &X &2 & \\ 
			\hline 
			\textbf{7. Ortografía} & El documento no muestra errores ortográficos. &2 &X &2 & \\ 
			\hline 
			\textbf{8. Madurez} & El Informe muestra de manera general una evolución de la madurez del código fuente,  explicaciones puntuales pero precisas y un acabado impecable.   (El profesor puede preguntar para refrendar calificación).  &4 &X &2 & \\ 
			\hline
			\multicolumn{2}{|c|}{\textbf{Total}} &20 & &15 & \\ 
			\hline
		\end{tabular}
		%\end{center}
		%\label{tab:multicol}
		}
	\end{table}
	
\clearpage

%------------------------------ REFERENCIAS ------------------------------

\section{Referencias}
\begin{itemize}			
	\item \url{https://docs.oracle.com/javase/tutorial/java/nutsandbolts/variables.html}
    \item \url{https://docs.oracle.com/javase/8/docs/api/java/util/Arrays.html}
\end{itemize}	
	
%\clearpage
%\bibliographystyle{apalike}
%\bibliographystyle{IEEEtranN}
%\bibliography{bibliography}
			
\end{document}